Part A: R preliminaries
Some quick remarks about R
\end{frame}
%======================================================================%
\begin{frame}[fragile]
\frametitle{ggplot2}
\large
\noindent \textbf{Preliminaries}
1. Inbuilt data sets
As part of the R installation, many inbuilt datasets are available for assisting R learners develop their knowledge using real data.
\end{frame}
%======================================================================%
\begin{frame}[fragile]
\frametitle{ggplot2}
\large
\noindent \textbf{Preliminaries}Some of the more commonly used data sets are iris and mtcars.
2. Dataframes
The conventional structure for a dataset is called a data frame. It is a table of rows and columns. Each row corresponds to a “case” and each column corresponds to a variable. Most packages including ggplot2 work best when the data is structured as a data frame.
A data frame can be accessed simply by typing in the name. mtcars
\end{frame}
%======================================================================%
\begin{frame}[fragile]
\frametitle{ggplot2}
\large
\noindent \textbf{Preliminaries}3. Inspecting a data set
Quite often a data set will be very large and it is not easy to get a sense of it by looking at it directly. There are some useful commands that give us important information about the data frame.
a. The \textbf{summary()} command
The \textbf{summary()} command is a very versatile command that gives the user a brief summary of the data object specified. summary(mtcars)
b. The \textbf{head()} command
The head command allows the user to read the column names and the data from the first six cases, allowing the user to develop a sense of the structure of the data set. head(mtcars)
c. The \textbf{dim()} command
The \textbf{dim()} command yields the numbers of rows and columns (i.e. the dimensions) of a data object. dim(mtcars)
d. The \textbf{names()} command
The \textbf{names()} command returns the columns names (i.e. variable names) of a data object. dim(mtcars)
\end{frame}
%======================================================================%
\begin{frame}[fragile]
\frametitle{ggplot2}
\large
\noindent \textbf{Preliminaries}4. Base R graphics
R has a default graphics packages installed automatically. Try out the following command to get a sense of it. plot(iris)
It is felt by many that the standard of graphical output required for professional use requires too much programming skill to be practicable for most users.
ggplot2 yields much better visualisations with much simpler code.
\end{frame}
%======================================================================%
\begin{frame}[fragile]
\frametitle{ggplot2}
\large
\noindent \textbf{Preliminaries}5. Installing and loading a package.
Packages can be added to the R workspace to provide greater functionality. 
Packages have to be downloaded from a mirror and then loaded into the workspace using the library() command. 
\begin{framed}
\begin{verbatim}
install.packages(“ggplot2”) 
library(ggplot2)
\end{verbatim}
\end{frame}
\end{frame}
%======================================================================%
\begin{frame}[fragile]
\frametitle{ggplot2}
\large
\noindent \textbf{Preliminaries}6. Colours
R supports a very wide range of colours. Type in the command colours() to see what there is available. colours()

\end{frame}
%======================================================================%
\begin{frame}[fragile]
\frametitle{ggplot2}
\large
\noindent \textbf{Preliminaries}


Part B: Introduction to ggplot2
\item “ggplot2” is a plotting system for R, based on the grammar of graphics, which tries to take the good parts of base and lattice graphics ( which are more established plotting systems in R) and none of the “bad” parts.
\item The package “ggplot2” was developed by Hadley Wickham, now an associate professor in Rice University, Texas.
\end{frame}
%======================================================================%
\begin{frame}[fragile]
\frametitle{ggplot2}
\large
Hadley Wickham has also developed other R packages such as
\item Reshape
\item plyr
\item lubridate
\item ggplot2 takes care of many of the fiddly details that make plotting a hassle (like drawing legends) as well as providing a powerful model of graphics that makes it easy to produce complex multi-layered graphics
\item ggplot2 is based on the book “Grammar of Graphics” by Leland Wilkinson (SPSS, previously Systax).
\item Essentially it is simple set of core principles on how to render graphics, that can be built up iteratively, and edited and re-edited later.
\item ggplot2 works in a layered fashioned, starting with the raw data at the base layer, and building up additional layers.
\end{frame}
%======================================================================%
\begin{frame}[fragile]
\frametitle{ggplot2}
\largeDocumentation and Websites
\item Wickham maintains a ggplot2 section on his website
o http://www.had.co.nz
o http://groups.google.com/groups/ggplot2
\item ggplot2 documentation is now available at docs.ggplot2.org.
What is the grammar of graphics
\item Wilkinson addresses the question of “what is a statistical graphic?”
\item Mapping from data to aesthetic attributes (e.g. colour, shape and size) of geometric objects (e.g. points, lines and bars)
\item Remark: geometric objects are referred to as “geoms”.
\item Wickham’s layered graphics approach is based on Wilkinson’s grammar, but adapts it to fit it into the R environment.
Other remarks
\item ggplot2 allows for convenient use of Faceting: a process where similar plots are generated for subsets of the data, and arranged so as to allow comparison between them.
\item Ggplot2 is for static graphics only. For interactive graphics, try out rggobi.
\end{frame}
%======================================================================%
\begin{frame}[fragile]
\frametitle{ggplot2}
\largePreliminaries
\item Installation is straightforward – performed in the same way as the installation of any R package.
\item Just remember to have recent version of R. >install.packages("ggplot2") >library(ggplot2)
\end{frame}
%======================================================================%
\begin{frame}[fragile]
\frametitle{ggplot2}
\large The ggplot2 Book
\item ggplot2 : Elegant Graphics r Data Analysis
\item Author: Hadley Wickham.
\item Springer-Verlag, New York, 2009.
\end{frame}
%======================================================================%
\begin{frame}[fragile]
\frametitle{ggplot2}
\large
Layered grammar
Layered grammar defines a plot as the combination of
\item A default data set and a setting of mappings from variables to aesthetics
\item One or more layers
\item One scale for each aesthetic mapping
\item A coordinate system
\item The faceting specification
\end{frame}
%======================================================================%
\begin{frame}[fragile]
\frametitle{ggplot2}
\largeLayers
Layers are responsible for creating the objects on the plot. Layers are composed of 4 parts
1) Data / aesthetic mapping
2) A statistical transformation (stat)
3) A geometric object (geom)
4) Position adjustment
\end{frame}
%======================================================================%
\begin{frame}[fragile]
\frametitle{ggplot2}
\large
Data sets in ggplot2
The following data sets are embedded in the ggplot2 package. We will mostly use “diamonds” and “mpg” for this workshop .
\item diamonds: Prices of 50,000 round cut diamonds
\item economics: US economic time series.
\item midwest: Midwest demographics.
\item movies: Movie information and user ratings from IMDB.com.
\item mpg: Fuel economy data from 1999 and 2008 for 38 popular models of car (very similar to mtcars)
\item msleep: An updated and expanded version of the mammals sleep dataset.
\item presidential: Terms of 10 presidents from Eisenhower to Bush W.
\item seals: Vector field of seal movements
I will also use the iris data set quite a bit too.
\end{frame}
