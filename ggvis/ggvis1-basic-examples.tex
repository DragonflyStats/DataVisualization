

\documentclass[MASTER.tex]{subfiles} 
\begin{document} 
%================================================================================= %
\begin{frame}[fragile]
\Large
\textbf{Recall:}\\
The \textbf{\textit{magrittr}} package allows you to rewrite the previous function call as:
\begin{framed}
\begin{verbatim}
mtcars %>%
        ggvis(x = ~wt, y = ~mpg) %>%
        layer_points()
\end{verbatim}
\end{framed}
\end{frame}
%================================================================================= %
\begin{frame}[fragile]
	\large
Pipe operator must be at the end of line, if using multiple lines
		\begin{framed}
			\begin{verbatim}
			mtcars %>%
			ggvis(x = ~wt, y = ~mpg) %>%
			layer_points()
			\end{verbatim}
		\end{framed}
This following code LOOKS neat, but doesn't work. 
		\begin{framed}
		\begin{verbatim}
		mtcars 
		    %>% ggvis(x = ~wt, y = ~mpg) 
		    %>% layer_points()
		\end{verbatim}
	\end{framed}
\end{frame}
%=================================================================================%
\begin{frame}
\frametitle{Data Visualization with ggvis}
\Large
\vspace{-1.8cm}
\begin{itemize}
% \item Don’t worry if this looks a little strange at first. You’ll soon get used to it!
\item This style of programming (i.e. using the pipe operator) also allows gives you a lot of power when you start creating a lot of power.
\item Also it allows you to seemlessly intermingle \textbf{\textit{ggvis}} and \textbf{\textit{dplyr}} code (\textit{Next Slide}).

\end{itemize}


\end{frame}
%==========================================================================%
\begin{frame}[fragile]
	\frametitle{Data Visualization with ggvis}
\Large	
\begin{framed}
\begin{verbatim}
library(dplyr)
# convert engine displacment to litres

mtcars %>%
  ggvis(x = ~mpg, y = ~disp) %>%
      mutate(disp = disp / 61.0237) %>% 
         layer_points()
\end{verbatim}
\end{framed}
\end{frame}

%==========================================================================%
\begin{frame}
\frametitle{Data Visualization with ggvis}

\Large
\textbf{Calling Formulas}
\begin{itemize}
\item The format of the visual properties needs a little explanation. 
\item We use $\sim$ before the variable name to indicate that we don’t want to literally use the value of the \textbf{mpg} variable (which doesn’t exist), but instead we want we want to use the \textbf{mpg} variable inside in the dataset. 
\item This is a common pattern in \textbf{ggvis}: we’ll always use formulas to refer to variables inside the dataset.

\end{itemize}
\end{frame}

%==========================================================================%
\begin{frame}[fragile]
	\frametitle{Data Visualization with ggvis}

\Large

The first two arguments to \texttt{ggvis()} are usually the position, so by convention you can drop \texttt{x} and \texttt{y}:
\begin{framed}
	\begin{verbatim}
mtcars   %>% 
  ggvis(~mpg, ~disp) %>% 
  layer_points()
\end{verbatim}
\end{framed}
($x$ for mpg, $y$ for displacement)
\end{frame}
%==========================================================================%
\begin{frame}[fragile]
	\Large
\textbf{All the \textit{mtcars} variables}

\begin{framed}
\begin{verbatim}
> names(mtcars)
[1] "mpg"  "cyl"  "disp" "hp"   "drat"
[6] "wt"   "qsec" "vs"   "am"   "gear"
[11] "carb"
> 
\end{verbatim}
\end{framed}

\end{frame}
%==========================================================================%
\begin{frame}[fragile]
	\frametitle{Data Visualization with ggvis}
\Large	
You can add more variables to the plot by mapping them to other visual properties like \textbf{fill, stroke, size} and \textbf{shape}.
{
	\large
\begin{framed}
	\begin{verbatim}
mtcars %>% 
  ggvis(~mpg, ~disp, stroke = ~vs) %>% 
     layer_points()
\end{verbatim}
\end{framed}
}

\end{frame}

%==========================================================================%
\begin{frame}[fragile]
	\frametitle{Data Visualization with ggvis}
		\Large

\textbf{The ``fill" property}		
\begin{framed}
	\begin{verbatim}
 mtcars %>% 
    ggvis(~mpg, ~disp, fill = ~vs) %>% 
       layer_points()
\end{verbatim}
\end{framed}

\end{frame}

%==========================================================================%
\begin{frame}[fragile]
	\frametitle{Data Visualization with ggvis}
		\Large
	\vspace{-1.4cm}
	\textbf{The ``size" property}		
\begin{framed}
	\begin{verbatim}	
 mtcars %>% 
    ggvis(~mpg, ~disp, size = ~vs) %>% 
        layer_points()
\end{verbatim}
\end{framed}

\end{frame}

%==========================================================================%
\begin{frame}[fragile]
	\frametitle{Data Visualization with ggvis}
	\Large
	
	\textbf{The ``shape" property}	
\begin{framed}
	\begin{verbatim}	
 mtcars %>% 
    ggvis(~mpg, ~disp, 
        shape = ~factor(cyl)) %>% 
    layer_points()
\end{verbatim}
\end{framed}

\end{frame}
	

%==========================================================================%
\begin{frame}
\frametitle{Data Visualization with ggvis}
\Large
 \textbf{The ``:=" operator}
\begin{itemize}
\item If you want to make the points a fixed colour or size, you need to use \texttt{:=} instead of \texttt{=}. 
\item The \texttt{:=} operator means to use a raw, unscaled value. 
\item This seems like something that \texttt{ggvis()} should be able to figure out by itself, but making it explicit allows you to create some useful plots that you couldn’t otherwise. 
% \item See the properties and scales for more details.
\end{itemize}


\end{frame}

%==========================================================================%
\begin{frame}[fragile]
	\frametitle{Data Visualization with ggvis}
	\Large
	
\begin{framed}
	\begin{verbatim}
mtcars  %>% 
   ggvis(~wt, ~mpg, fill := "red", 
         stroke := "black") %>% 
      layer_points()
\end{verbatim}
\end{framed}

\end{frame}

%==========================================================================%
\begin{frame}[fragile]
	\frametitle{Data Visualization with ggvis}
	\Large
	
\begin{framed}
	\begin{verbatim}
mtcars   %>% 
     ggvis(~wt, ~mpg, 
          size := 300, 
          opacity := 0.4)    %>% 
     layer_points()
\end{verbatim}
\end{framed}

\end{frame}

%==========================================================================%
\begin{frame}[fragile]
	\frametitle{Data Visualization with ggvis}	
	\Large
	
\begin{framed}
	\begin{verbatim}
mtcars  %>% 
    ggvis(~wt, ~mpg, 
        shape := "cross")   %>% 
    layer_points()
\end{verbatim}
\end{framed}

\end{frame}

\end{document}
%====================================================================================== %