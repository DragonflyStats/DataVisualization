\documentclass[MASTER.tex]{subfiles} 
\begin{document} 
%================================================= %	
\begin{frame}
\frametitle{Aesthetics for ggvis}
\Large
As with all graphics there are a number of
aesthetics we can set
\begin{itemize}
\item stroke
\item fill
\item size
\item opacity
\end{itemize}
\end{frame}

%=================================================%
\begin{frame}
\frametitle{Changing Aesthetics based on variables}
\Large
\begin{itemize}
\item In ggvis we map a variable to a property using \texttt{=}
\item We have to remember to use the "$\sim$" with all
variable names
\item \texttt{fill = }$\sim$ \texttt{Line} would set the fill based on the Line
variable
\end{itemize}
\end{frame}
%=================================================%
\begin{frame}[fragile]
\Large
\begin{framed}
	\begin{verbatim}
tubeData %>%
   ggvis(x = ~Month, y = ~Excess, 
        fill = ~Type) %>%
      layer_points()
\end{verbatim}
\end{framed}

\end{frame}
%=================================================%
\begin{frame}[fragile]
\frametitle{Setting property values}
\vspace{-1.9cm}
\Large
\begin{itemize}
\item When we set a property based on a value we use
\texttt{":="}
\item \texttt{fill := "red"} would set the fill to red
\end{itemize}
\end{frame}
%=================================================%
\begin{frame}[fragile]
	\Large
\begin{framed}
\begin{verbatim}
tubeData %>%
   ggvis(x = ~Month, y = ~Excess, 
         fill := "orange",
         opacity := 0.6) %>%
       layer_points()

\end{verbatim}
\end{framed}


\end{frame}
%=================================================%
\begin{frame}
\frametitle{Exercise}
\Large
\textbf{Exercise}
\begin{itemize}
\item Create a plot of mpg against wt using the mtcars
data
\item Update the plot to colour by the cylinder variable,
ensure that the points are coloured by distinct
colours rather than on a scale
\item Update the plotting symbol to be triangles.
\end{itemize}
\end{frame}
%=================================================%
\end{document}