 
  plotly Sign In SIGN UP  + NEW PROJECT  UPGRADE  REQUEST DEMO Feed Pricing Dashboards API
Plotly for R is now entirely open source, free, and self-hosted
Learn more about why we've open sourced

 HELP/API LIBRARIES/R/ FORK ON GITHUB
QUICK START
Getting Started
Chart Attributes
User Guide
GGPlot2
Shiny Tutorial
Use Offline
EXAMPLES
Basic
Statistical
Scientific
Maps
3D
Events and Controls
Layout Options
File Settings
Image Export & Retrieving Plots
COMMUNITY
GitHub
community.plot.ly
TOOLS
Web Editor
Plotly R Library 2.0
Plotly for R is an interactive, browser-based charting library built on the open source JavaScript graphing library, plotly.js. It works entirely locally, through the HTML widgets framework.
library(plotly)
set.seed(100)
d <- diamonds[sample(nrow(diamonds), 1000), ]
plot_ly(d, x = carat, y = price, text = paste("Clarity: ", clarity),
        mode = "markers", color = carat, size = carat)

Plotly graphs are interactive. Click-drag to zoom, shift-click to pan, double-click to autoscale.

KNOW AND LOVE GGPLOT2? TRY GGPLOTLY

p <- ggplot(data = d, aes(x = carat, y = price)) +
  geom_point(aes(text = paste("Clarity:", clarity)), size = 4) +
  geom_smooth(aes(colour = cut, fill = cut)) + facet_wrap(~ cut)

(gg <- ggplotly(p))

MIX DATA MANIPULATION AND VISUALIZATION VERBS

Plotly objects are data frames with a class of plotly and an environment that tracks the mapping from data to visual properties.

str(p <- plot_ly(economics, x = date, y = uempmed))
## Classes 'plotly' and 'data.frame':   478 obs. of  6 variables:
##  $ date    : Date, format: "1967-06-30" "1967-07-31" ...
##  $ pce     : num  508 511 517 513 518 ...
##  $ pop     : int  198712 198911 199113 199311 199498 199657 199808 199920 200056 200208 ...
##  $ psavert : num  9.8 9.8 9 9.8 9.7 9.4 9 9.5 8.9 9.6 ...
##  $ uempmed : num  4.5 4.7 4.6 4.9 4.7 4.8 5.1 4.5 4.1 4.6 ...
##  $ unemploy: int  2944 2945 2958 3143 3066 3018 2878 3001 2877 2709 ...
##  - attr(*, "plotly_hash")= chr "7ff330ec8c566561765c62cbafed3e0f#2"
This allows us to mix data manipulation and visualization verbs in a pure(ly) functional, predictable and pipeable manner. Here, we take advantage of dplyr's filter() verb to label the highest peak in the time series:

p %>%
  add_trace(y = fitted(loess(uempmed ~ as.numeric(date)))) %>%
  layout(title = "Median duration of unemployment (in weeks)",
         showlegend = FALSE) %>%
  dplyr::filter(uempmed == max(uempmed)) %>%
  layout(annotations = list(x = date, y = uempmed, text = "Peak", showarrow = T))

3D WEBGL AND MORE

Although data frames can be thought of as the central object in this package, plotly visualizations don't actually require a data frame. This makes chart types that accept a z argument especially easy to use if you have a numeric matrix:

plot_ly(z = volcano, type = "surface")

RUN LOCALLY OR PUBLISH TO THE WEB

By default, plotly for R runs locally in your web browser or R Studio's viewer. You can publish your graphs to the web by creating a plotly account.

Plotly hosting is free for public charts. If you have sensitive data, upgrade to a paid plan.

library(plotly)
p <- plot_ly(midwest, x = percollege, color = state, type = "box")
# plotly_POST publishes the figure to your plotly account on the web
plotly_POST(p, filename = "r-docs/midwest-boxplots", world_readable=TRUE)

Documentation Examples
BASIC CHARTS


 Dashboards
Dashboards


 Line and Scatter Plots
Line and Scatter Plots


 Bar Charts
Bar Charts


 Filled Area Plots
Filled Area Plots


 Dumbbell plots
Dumbbell plots


 Dot plots
Dot plots


 Bubble Charts
Bubble Charts


 Shapes
Shapes


 Graphing Multiple Chart Types
Graphing Multiple Chart Types


 Gauge Chart
Gauge Chart

STATISTICAL CHARTS


 Histograms
Histograms


 Box Plots
Box Plots


 2D Histograms
2D Histograms


 Error Bars
Error Bars

SCIENTIFIC CHARTS


 Network Graph
Network Graph


 Contour Plots
Contour Plots


 Heatmaps
Heatmaps


 Polar Charts
Polar Charts

MAPS


 Map Subplots And Small Multiples
Map Subplots And Small Multiples


 Scatter Plots on Maps
Scatter Plots on Maps


 Bubble Maps
Bubble Maps


 Choropleth Maps
Choropleth Maps


 Lines on Maps
Lines on Maps

3D CHARTS


 3D Line Plots
3D Line Plots


 3D Scatter Plots
3D Scatter Plots


 3D Surface Plots
3D Surface Plots

ADD CUSTOM CONTROLS WITH JAVASCRIPT

All Plotly charts have click, hover and zoom events exposed to add custom controls with Plotly's JavaScript postMessage API.


 Click Events
Click Events


 Hover Events
Hover Events


 Button Events
Button Events


 Adding Zoom Effects
Adding Zoom Effects


 Dropdown Events
Dropdown Events


 Adding Sliders to Charts
Adding Sliders to Charts

LAYOUT OPTIONS

Text and Annotations
Subplots
Setting Graph Size
Multiple Axes
Legends
Font
Axes Labels
Text and Annotations
Axes
LaTeX Typesetting in R Graphs
FILE SETTINGS

Using Plotly with knitr
Public vs Private Graphs
Filenames, Folders, and Updating Plotly Graphs in R
Sending data to charts
IMAGE EXPORT & RETRIEVING PLOTS

Getting Started with Shiny and Plotly
Get Requests
upgrade to pro
unlimited private files & email support
$20 PER MONTH
API

Documentation
API Libraries
REST APIs
Plotly.js
Hardware
ABOUT US

Team
Careers
Plotly Blog
Modern Data
HELP

Knowledge Base
Benchmarks
Workshop
SOLUTIONS

Plans & Pricing
Enterprise
Education
Plotly.js
CONNECT

    
Copyright © 2015 Plotly. All rights reserved.

Terms of Service

Privacy Policy
%=======================================================================%
 plotly Sign In SIGN UP  + NEW PROJECT  UPGRADE  REQUEST DEMO Feed Pricing Dashboards API
Plotly for R is now entirely open source, free, and self-hosted
Learn more about why we've open sourced

 HELP/API LIBRARIES/R/GETTING STARTED WITH PLOTLY FOR R FORK ON GITHUB
Getting Started with Plotly for R
Plotly is an R package for creating interactive web-based graphs via the open source JavaScript graphing library plotly.js. As of version 2.0 (November 17, 2015), Plotly graphs are rendered locally through the htmlwidgets framework.

 Build Status

INSTALLATION

Plotly is now on CRAN!

install.packages("plotly")

Or install the latest development version (on GitHub) via devtools:

# install.packages("devtools")
devtools::install_github("ropensci/plotly")

RStudio users should download the latest RStudio release for compatibility with htmlwidgets.

SIMPLE EXAMPLE

library(plotly)
p <- plot_ly(midwest, x = percollege, color = state, type = "box")
p
Simply printing the plotly object will render the chart locally in your web browser or in the R Studio viewer.


Plotly graphs are interactive. Click on legend entries to toggle traces, click-and-drag on the chart to zoom, double-click to autoscale, shift-and-drag to pan.

HOSTING GRAPHS IN YOUR ONLINE PLOTLY ACCOUNT

By default, plotly for R runs locally in your web browser or in the R Studio viewer. You can publish your charts to the web with plotly's web service.

1 - Create a free plotly account

A plotly account is required to publish charts online. It's free to get started, and you control the privacy of your charts.

2 - Save your authentication credentials

Find your authentication API keys in your online settings. Set them in your R session with:

Sys.setenv("plotly_username"="your_plotly_username")
Sys.setenv("plotly_api_key"="your_api_key")
Save these commands in your .Rprofile file to be run everytime you start R.

3 - Publish your graphs to plotly with plotly_POST

library(plotly)
p <- plot_ly(midwest, x = percollege, color = state, type = "box")
plotly_POST(p, filename = "r-docs/midwest-boxplots", world_readable=TRUE)
filename sets the name of the file inside your online plotly account.

world_readable sets the privacy of your chart. If TRUE, the graph is publically viewable, if FALSE, only you can view it.

SPECIAL INSTRUCTIONS FOR PLOTLY ON-PREMISE USERS

Your API key for account on the public cloud will be different than the API key in Plotly On-Premise. Visit https://plotly.your-company.com/settings/api/ to find your Plotly On-Premise API key. Remember to replace "your-company.com" with the URL of your Plotly On-Premise server.

If your company has a Plotly On-Premise server, change the R API endpoint so that it points to your company's Plotly server instead of Plotly's cloud.

In your .RProfile write:

Sys.setenv("plotly_domain"="https://plotly.your-company.com")  
Sys.setenv("plotly_api_domain" = "https://api-plotly.your-company.com")
Remember to replace "your-company" with the URL of your Plotly On-Premise server.

MAKE YOUR FIRST GRAPH
upgrade to pro
unlimited private files & email support
$20 PER MONTH
API

Documentation
API Libraries
REST APIs
Plotly.js
Hardware
ABOUT US

Team
Careers
Plotly Blog
Modern Data
HELP

Knowledge Base
Benchmarks
Workshop
SOLUTIONS

Plans & Pricing
Enterprise
Education
Plotly.js
CONNECT

    
Copyright © 2015 Plotly. All rights reserved.

Terms of Service

Privacy Policy

