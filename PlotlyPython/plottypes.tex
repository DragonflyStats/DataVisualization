%=========================================================%

\texttt{Start Plotting Online}
\begin{itemize}
	\item When plotting online, the plot and data will be saved to your cloud account. 
	\item There are two methods for plotting online: py.plot() and py.iplot(). Both options create a unique url for the plot and save it in your Plotly account.
	\item Use py.plot() to return the unique url and optionally open the url.
	\item Use py.iplot() when working in a Jupyter Notebook to display the plot in the notebook.
\end{itemize}

%=========================================================%

In [5]:
import plotly.plotly as py
from plotly.graph_objs import *

trace0 = Scatter(
    x=[1, 2, 3, 4],
    y=[10, 15, 13, 17]
)
trace1 = Scatter(
    x=[1, 2, 3, 4],
    y=[16, 5, 11, 9]
)
data = Data([trace0, trace1])

py.plot(data, filename = 'basic-line')
%=======================%
High five! You successfuly sent some data to your account on plotly. 

%=======================%
View your plot in your browser at https://plot.ly/~DemoAccount/0 or inside your plot.ly 
account where it is named 'basic-line'
%=======================%
Out[5]:
u'https://plot.ly/~DemoAccount/0'
%=======================%
Checkout the docstrings for more information:
In [6]:
import plotly.plotly as py
help(py.plot)
%=======================%
Help on function plot in module plotly.plotly.plotly:
<LIVE DEMO>

%=======================%
\begin{frame}[fragile]

Local Plot - rendered on jupyter notebook
\begin{framed}
\begin{verbatim}

import plotly.plotly as py
from plotly.graph_objs import *

trace0 = Scatter(
    x=[1, 2, 3, 4],
    y=[10, 15, 13, 17]
)
trace1 = Scatter(
    x=[1, 2, 3, 4],
    y=[16, 5, 11, 9]
)
data = Data([trace0, trace1])

py.iplot(data, filename = 'basic-line')
%=======================%
Out[9]:

See more examples in our IPython notebook documentation.
Check out the py.iplot() docstring for more information.
%=======================%
